\chapter*{Résumé}

Dans ce rapport de stage de fin d'étude, on aborde la complexité des activités cognitives lors de la compréhension de phrases avec une démarche inter-disciplinaire des systèmes complexes en nous référant principalement aux formalismes de la théorie de l'information et des systèmes dynamiques \cite{8}. Dans ce contexte, on explore la compréhension linguistique sur la base de mesures de signaux neurophysiologiques, issus de l'activité cérébrale, collectés à partir d'un magnétoencéphalographe (MEG). Ces mesure permettre de capturer la dynamique cérébrale que l'on cherche à indexer à différentes conditions expérimentales, qui reflètent difféents degrés de complexité linguistique.

Sur le plan méthodologique, on a utilisé une approche basée sur la représentation symbolique d'un système dynamique. Ces méthodes permettent de quantifier la complexité intrinsèque de la dynamique cérébrale et la quantité d'information produite lors de la compréhension linguistique.
