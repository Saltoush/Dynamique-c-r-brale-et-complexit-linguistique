\chapter*{Conclusion}
\addcontentsline{toc}{chapter}{Conclusion}

A travers le travail que j'ai mené durant mon stage, j'ai pu appliquer une démarche méthodologique basée sur la théorie de l'information et la théorie des systèmes dynamiques. Grâce à la représentation symbolique et sur la base de signaux MEG qui constituent une mesure de l'activité cérébrale, j'ai réussi mettre en exergue le lien entre dynamique cérébrale et complexité linguistique en utilisant comme métrique l'entropie. En effet, dans le contexte où on l'a utilisé, la représentation symbolique permet la majeure partie du temps de caractériser un système dynamique continu (le cerveau) comme une séquence de symboles correspondants à des états discrets. Cela permet de rendre compte de la complexité intrinsèque de la dynamique cérébrale et de la quantité d'information contenue dans celle-ci grâce à l'entropie. Le taux d'entropie comme résultat final de l'algorithme mis en place durant le stage semble être un bon indicateur de la complexité linguistique et a permis de discriminer des conditions expérimentales à partir des mesures de l'activité cérébrale. 

Comme perspectives futures, il serait intéressant d'appliquer une segmentation temporelle différente et s'intéresser à la variation du taux d'entropie au cours d'une phrase. Aussi, il faudrait appliqué l'algorithme développé au cours du stage sur l'ensemble des 204 sujets de la base de données MOUS. En effet, de par la variabilité interindividuelle, il serait intéressant de mener l'étude sur une plus grande quantité de sujets différents. Cela permettrait de réduire encore plus le rapport signal sur bruit et obtenir des résultats plus robustes. On notera aussi qu'il existe des méthodes de représentation symbolique plus complexes qui permettent de rendre compte plus en détails de la dynamique cérébrale. En effet, la décomposition atomique \cite{7} est une méthode de représentation symbolique qui consiste à extraire des atomes spatio-temporels du signal. Cette manière plus fine de représenter symboliquement un système dynamique pourrait être une des pistes de développement futur du projet.
